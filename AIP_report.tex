\documentclass[12pt,a4j]{jsarticle}

\usepackage{bm}	% ベクトル修飾の太字がキレイに出せる
\usepackage{mathrsfs}	% 飾文字
\usepackage{amsmath,amssymb} % 数学系

\newcommand{\xa}{x_\alpha}
\newcommand{\ya}{y_\alpha}
\newcommand{\xia}{\bm{\xi}_\alpha}

\title{画像工学特論 レポート}
\author{163377 宮田木織}

\begin{document}
  \maketitle % タイトルを出力

\section*{課題2}
  式を以下のように置く。

  \begin{align}
    \xia &= \begin{pmatrix}\xa^2 & 2\xa\ya & \ya^2 & 2f_0\ya & f_0^2\end{pmatrix}^\top \\
    \bm{u} &= \begin{pmatrix}A & B & C & D & E & F\end{pmatrix}^\top
  \end{align}

  \subsection*{2-1}

  全ての$\xia$が誤差を含まないとすると、
  
\end{document}