\documentclass[12pt,a4j]{jsarticle}

\usepackage{bm}	% ベクトル修飾の太字がキレイに出せる
\usepackage{mathrsfs}	% 飾文字
\usepackage{amsmath,amssymb} % 数学系

\newcommand{\xa}{x_\alpha}
\newcommand{\ya}{y_\alpha}
\newcommand{\xia}{\bm{\xi}_\alpha}

\title{画像工学特論 レポート}
\author{163377 宮田木織}

\begin{document}
  \maketitle % タイトルを出力

\section*{課題2}
  式と式番号を以下のように置きなおす。

  \begin{align}
    \xia = \begin{pmatrix}\xa^2 & 2\xa\ya & \ya^2 & 2f_0\ya & f_0^2\end{pmatrix}^\top
     & \quad\cdots\text{楕円に対する誤差を含む点データ} \\
    \bar{\xia} = \begin{pmatrix}\bar{\xa}^2 & 2\bar{\xa}\bar{\ya} & \bar{\ya}^2 & 2f_0\bar{\ya} & f_0^2\end{pmatrix}^\top
     & \quad\cdots\text{$\xa$と$\ya$を真値とした$\xia$} \\
    \bm{u} = \begin{pmatrix}A & B & C & D & E & F\end{pmatrix}^\top
     & \quad\cdots\text{楕円のパラメータ} \\
    \bm{M} = \sum_{\alpha = 1}^N \xia \xia^\top
     & \quad\cdots\text{レポート問題文中式(2)より} \label{eq:M}
  \end{align}

  \subsection*{2-1}

  全ての$\bar{\xia}$について下式が成り立つ。

  \begin{equation}
    (\bm{u}, \bar{\xia}) = 0
  \end{equation}

  ここから単純に考えると、$\xia$に関しては、下式の$(\bm{u}, \xia)$の2乗和$J_{LS}$が最小となるような$\bm{u}$がパラメータとして最もらしいと言える。

  \begin{equation}
    J_{LS} = \sum_{\alpha = 1}^N (\bm{u}, \xia)^2
  \end{equation}

  最小二乗法は、この考えに基づき、$J_{LS}$を目的関数として最小化することで、$\bm{u}$を推測する手法である。\par
  ここで、この式を展開・整理すると下式となる。

  \begin{align}
    \sum_{\alpha = 1}^N (\bm{u}, \xia)^2 &= \sum_{\alpha = 1}^N (\xia, \bm{u})^2 \nonumber \\
     &= \sum_{\alpha = 1}^N \xia^\top \bm{u} \xia^\top \bm{u} \nonumber \\
     &= \sum_{\alpha = 1}^N \bm{u}^\top \xia \xia^\top \bm{u} \nonumber \\
     &= \bm{u}^\top \left(\sum_{\alpha = 1}^N \xia \xia^\top \right) \bm{u} \label{eq:organized}
  \end{align}

  式(\ref{eq:organized})を式(\ref{eq:M})を用いて置きなおすことで、$J_{LS}$は以下のように書くことができる。

  \begin{equation}
    J_{LS} = (\bm{u}, \bm{Mu})
  \end{equation}

  以上のとおり、最小二乗法で楕円のパラメータを推定するときの目的関数$J_{LS}$が示された。
\end{document}